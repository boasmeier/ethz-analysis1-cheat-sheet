\mysection[Rhodamine]{\centering Beweisideen}
\DEF{Limes Berechnen}{\begin{enumerate}
    \item Polynome: In Polynomform bringen, dann grösste Potenz weg kürzen.
    \item Wurzelfunktionen: Rationalisieren des Zählers oder Nenners.
    \item Exponenten: $x\mapsto e^{ln(x)}$-Trick.
    \item Brüche: Wenn Zähler und Nenner beide gegen $0,-\infty,\infty$ konvergieren, verwende l'Hospital.
\end{enumerate}}

\DEF{Injektivität}{
\begin{enumerate}
    \item Zeige $f(x_1)=f(x_2)\Rightarrow x_1=x_2$.
    \item Annehmen, dass $\exists\  x_1\not =x_2:f(x_1)=f(x_2)$. Dann Wiederspruch herbeiführen.
    \item Zeige strenge Monotonie.
\end{enumerate}
}

\DEF{Surjektivität}{
\begin{enumerate}
    \item Zeige Stetigkeit und, dass Maximum und Minimum den Ergebnisbereich einschliessen.
    \item Zeige dass $f^{-1}$ keine Lücken im Definitionsbereich hat.
\end{enumerate}
}

\DEF{Bijektivität}{
\begin{enumerate}
    \item Zeige Injektivität und Surjektivität.
\end{enumerate}}

\DEF{Monotonie}{\begin{enumerate}
    \item Erste Ableitung
    \item Zweite Ableitung
    \item Definition der Monotonie
    \item Folgen
    \item Differenzquotient
    \item Integraltest
    \item Direktes Ablesen vom Funktionsausdruck
\end{enumerate}}

\DEF{Stetigkeit}{
\begin{enumerate}
    \item Prüfe ob es eine bereits bekannte Funktion ist.
    \item Komposition, Summe, Produkt, Division stetiger Funktionen ist ebenfalls stetig.
\end{enumerate}}

\DEF{Glattheit (Smoothness)}{
\begin{enumerate}
    \item Prüfe ob es eine bereits bekannte Funktion ist.
    \item Komposition, Summe, Produkt, Division glatter Funktionen ist ebenfalls glatt.
\end{enumerate}
}

\DEF{Konvergenz von Folgen zeigen}{\begin{enumerate}
    \item Limes berechnen
    \item Sandwichlemma
    \item Weierstrass
    \item Cauchy Kriterium
\end{enumerate}}

\DEF{Divergenz von Folgen zeigen}{
\begin{enumerate}
    \item Finde zwei Teilfolgen mit ungleichen Grenzwerten(z.B. grade/ungerade).
\end{enumerate}}

\DEF{Konvergenz von Reihen zeigen}{\begin{enumerate}
    \item $\lim_{n\rightarrow\infty}a_n\not=0\Rightarrow$ divergent.
    \item Vergleichssatz
    \item Quotientenkriterium
    \item Wurzelkriterium
    \item Leibnizkriterium
    \item Integral-Test
\end{enumerate}}

\DEF{Konvergenz Uneigentlicher Integrale zeigen}{\begin{enumerate}
    \item Vergleichssatz
    \item Falls Integrationsbereich nicht $\infty$: Grenzwert der nicht definierten Endpunkten des Integrationsbereichs prüfen. Wenn Grenzwert existiert $\Rightarrow$ konvergent.
    \item Limes berechnen \begin{enumerate}
        \item Stammfunktion des unbestimmten Integrals finden.
        \item Basierend auf unbestimmtem Integral, Stammfunktion des bestimmten Integrals von $0$ bis $b$ finden.
        \item Limes berechnen.
    \end{enumerate}
\end{enumerate}}