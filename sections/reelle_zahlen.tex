\mysection[Orchid]{\centering Reelle Zahlen und Euklidische Räume}
\DEF{Natürliche Zahlen}{$\mathbb{N}=\{0,1,2,...\}$. $x+1=0$ hat in $\mathbb{N}$ keine Lösung. $\mathbb{N}^*=\{1,2,3,...\}$.}

\DEF{Ganze Zahlen}{$\mathbb{Z}=\{...,-2,-1,0,1,2,...\}$. $2x=1$ hat in $\mathbb{Z}$ keine Lösung.}

\DEF{Rationale Zahlen}{$\mathbb{Q}=\{\frac{p}{q}|p,q\in\mathbb{Z},q\neq0\}$. $c^2=a^2+b^2$ oder Kreisumfang $L=2\pi$ haben in $\mathbb{Q}$ keine Lösung.}

\SA{1.1.1 Lindemann 1882}{Es gibt keine Gleichung der Form $x^n+a_{n-1}x^{n-1}+...+a_0=0$ mit $a_i\in\mathbb{Q}$, s.d. $x=\pi$ eine Lösung ist.}

\mysubsection{Körper der Reellen Zahlen}
\DEF{Operationen}{

Addition: $+:\mathbb{R}\times\mathbb{R}\rightarrow\mathbb{R},(x,y)\mapsto x+y$.

Multiplikation: $\cdot:\mathbb{R}\times\mathbb{R}\rightarrow\mathbb{R},(x,y)\mapsto x\cdot y$.}

\SA{1.1.2 Ordnungsrelation}{$\leq$ für welche gilt $\mathbb{R}$ ist ein kommutativer, angeordneter Körper, der ordnungsvollständig ist.}

\DEF{Axiome der Addition}{
\begin{enumerate}
    \item[A1] Assoziativität: $\forall x,y,z \in\mathbb{R}: x+(y+z)=(x+y)+z$ 
    \item[A2] Neutrales Element: $\forall x\in\mathbb{R}: x+0=x$
    \item[A3] Inverses Element: $\forall x\in\mathbb{R},\exists y\in\mathbb{R}: x+y=0 \Rightarrow y=-x$
    \item[A4] Kommutativität: $\forall x,y\in\mathbb{R}: x+z=z+x$
\end{enumerate}}

\DEF{Axiome der Multiplikation}{
\begin{enumerate}
    \item[M1] Assoziativität: $\forall x,y,z \in\mathbb{R}: x\cdot(y\cdot z)=(x\cdot y)\cdot z$
    \item[M2] Neutrales Element: $\forall x\in\mathbb{R}: x\cdot 1=x$
    \item[M3] Inverses Element: $\forall x\in\mathbb{R}, x\neq0,\exists y\in\mathbb{R}: x\cdot y=1\Rightarrow y=x^{-1}=\frac{1}{x}$
    \item[M4] Kommutativität: $\forall x,y\in\mathbb{R}: x\cdot z=z\cdot x$
\end{enumerate}
}

\DEF{Distributivität}{$\forall x,y,z\in\mathbb{R}: x\cdot (y+z)= x\cdot y+x\cdot z$}

\DEF{Ordnungsaxiome}{
\begin{enumerate}
    \item[O1] Reflexivität: $\forall x\in\mathbb{R}: x\leq x$
    \item[O2] Transitivität: $(x\leq y \land y\leq z)\Rightarrow x\leq z$
    \item[O3] Antisymmetrie: $(x\leq y \land y\leq x)\Rightarrow x=y$
    \item[O4] Total: $\forall x,y\in\mathbb{R}: x\leq y \lor y\leq x$
\end{enumerate}}

\DEF{Kompatibilität}{
\begin{enumerate}
    \item[K1] $\forall x,y,z\in\mathbb{R}:x\leq y \Rightarrow x+z\leq y+z$
    \item[K2] $\forall x\geq0,$ $\forall y\geq0:x\cdot y\geq 0$
\end{enumerate}}

\DEF{Ordnungsvollständigkeit}{Seien $A,B\subseteq\mathbb{R}$ s.d. 
\begin{enumerate}
    \item[i.] $A\neq\emptyset, B\neq\emptyset$,
    \item[ii.] $\forall a\in A,\forall b\in B:a\leq b$.
\end{enumerate}
Dann $\exists c\in\mathbb{R}$ s.d. $(\forall a\in A:a\leq c) \land (\forall b\in B: c\leq b)$.

$\mathbb{Q}$ ist nicht ordnungsvollständig, weil es in $\mathbb{Q}$ Folgen gibt, die gegen nicht rationale Zahlen wie z.B. $\sqrt{2}$ konvergieren. $\mathbb{Q}$ ist somit lückenhaft und erfüllt obige Definition nicht.}

\COR{1.1.6}{Folgerungen obiger Axiome:
\begin{enumerate}
    \item Additive und multiplikative Inverse sind eindeutig.
    \item $\forall x\in\mathbb{R}:0\cdot x=0$
    \item $\forall x\in\mathbb{R}:(-1)\cdot x=-x \Rightarrow (-1)^2=1,-(x\cdot y)=(-x)\cdot (-y)=x\cdot y$
    \item $y\geq 0 \Leftrightarrow (-y)\leq 0$
    \item $\forall y\in\mathbb{R}: y^2\geq 0 \Rightarrow 1^2=1\cdot 1=1\geq 0$.
    \item $(x\leq y \land u\leq v) \Rightarrow x+u \leq y+v$
    \item $(0\leq x \leq y) \land (0\leq u \leq v )\Rightarrow x\cdot u \leq y\cdot v$
    \item $\forall x,y\in\mathbb{R}: 0<x\leq y \Leftrightarrow 0<y^{-1}\leq x^{-1}$
\end{enumerate}}

\COR{1.1.7 Archimedisches Prinzip}{Sei $x\in\mathbb{R}$ mit $x>0$ und $y\in\mathbb{R}$. Dann  $\exists n\in\mathbb{N}: y\leq n\cdot x$.}

\SA{1.1.8}{$\forall t\geq 0,t\in\mathbb{R}$ hat $x^2=t$ eine Lösung in $\mathbb{R}$. Sie wird mit $\sqrt{t}$ bezeichnet.}

\DEF{Absolutbetrag}{Seien $x,y\in\mathbb{R}$. \begin{enumerate}
    \item $max\{x,y\}=\begin{cases}
        x & \text{if $y\leq x$} \\
        y & \text{if $x\leq y$} 
    \end{cases}$
    \item $min\{x,y\}=\begin{cases}
        x & \text{if $x\leq y$} \\
        y & \text{if $y\leq x$} 
    \end{cases}$
    \item $x\in\mathbb{R}: |x|=max\{x,-x\}$.
\end{enumerate}}

\SA{1.1.10 Rechenregeln für Absolutbetrag}{
\begin{enumerate}
    \item $|x|\geq 0\ \forall x\in\mathbb{R}$
    \item $|xy|=|x||y|\ \forall x,y\in\mathbb{R}$
    \item $|x\pm y|\leq |x|+|y|\ \forall x,y\in\mathbb{R}$ (Dreiecksungl.)
    \item $|x\pm y|\geq ||x|-|y||\ \forall x,y\in\mathbb{R}$
\end{enumerate}}

\SA{1.1.11 Young'sche Ungleichung}{$$\forall\varepsilon>0,\forall x,y\in\mathbb{R}: 2|xy|\leq \varepsilon x^2 + \frac{1}{\varepsilon}y^2$$}

\DEF{Unendlich}{$$\forall x\in\mathbb{R}: -\infty < x < +\infty \Leftrightarrow -\infty \not\in\mathbb{R} \land +\infty\not\in\mathbb{R}$$}

\DEF{Intervall}{Ein Intervall ist eine Teilmenge von $\mathbb{R}$ von der Form:
\begin{enumerate}
    \item für $a\leq b \in \mathbb{R}:$
    \begin{align*}
        [a,b]&:=\{x\in\mathbb{R}|a\leq x \leq b\},\\
        [a,b)=[a,b[&:=\{x\in\mathbb{R}|a\leq x < b\},\\
        (a,b]=]a,b]&:=\{x\in\mathbb{R}|a< x \leq b\},\\
        (a,b)=]a,b[&:=\{x\in\mathbb{R}|a< x < b\}
    \end{align*}
    \item für $a\in\mathbb{R}:$
    \begin{align*}
        [a,+\infty)=[a,+\infty[&:=\{x\in\mathbb{R}|a\leq x\},\\
        ]a,+\infty)=]a,+\infty[&:=\{x\in\mathbb{R}|a < x\},\\
        (-\infty,a]=]-\infty,a]&:=\{x\in\mathbb{R}|a \geq x\},\\
        (-\infty,a)=]-\infty,a[&:=\{x\in\mathbb{R}|a > x\}
    \end{align*}
    \item $(-\infty,+\infty)=]-\infty,+\infty[ = \mathbb{R}$
\end{enumerate}}

\DEF{Intervall Begrifflichkeiten}{\begin{itemize}
    \item Abgeschlossen, kompakt: $[a,b]$
    \item Abgeschlossen, nicht kompakt $[a,+\infty), (-\infty,a]$
    \item Halboffen $[a,b),(a,b]$
    \item Offen: $(a,b),(a,+\infty),(-\infty,a)$
\end{itemize}}

\DEF{Obere Schranke}{Sei $A\subseteq\mathbb{R}$. $c\in\mathbb{R}$ ist eine obere Schranke von $A \Leftrightarrow$ $\forall a\in A: a\leq c \Leftrightarrow $ $A$ ist nach oben beschränkt.}

\DEF{Untere Schranke}{Sei $A\subseteq\mathbb{R}$. $c\in\mathbb{R}$ ist eine untere Schranke von $A \Leftrightarrow$ $\forall a\in A: c\leq a \Leftrightarrow $ $A$ ist nach unten beschränkt.}

\DEF{Beschränkt}{Sei $A\subseteq\mathbb{R}$. $A$ heisst beschränkt $\Leftrightarrow A$ hat untere und obere Schranke.}

\DEF{Maximum}{Sei $A\subseteq\mathbb{R}$. $m\in\mathbb{R}$ ist ein Maximum von $A \Leftrightarrow (m\in A) \land (m$ ist obere Schranke von $A)$.}

\DEF{Minimum}{Sei $A\subseteq\mathbb{R}$. $m\in\mathbb{R}$ ist ein Minimum von $A \Leftrightarrow (m\in A) \land (m$ ist untere Schranke von $A)$.}

\DEF{Supremum}{Sei $A\subseteq\mathbb{R}, A\not = \emptyset$. Sei $A$ nach oben beschränkt. Dann $\exists$ kleinste obere Schranke von $A$: $c:=sup(A)$ genannt Supremum und $[sup(A),\infty)$ ist die Menge aller oberen Schranken von $A$. $A$ nicht nach oben beschränkt $\Leftrightarrow sup(A) = \infty$. $sup(\emptyset)=\infty$.}

\DEF{Infimum}{Sei $A\subseteq\mathbb{R}, A\not = \emptyset$. Sei $A$ nach unten beschränkt. Dann $\exists$ grösste untere Schranke von $A$: $d:=inf(A)$ gennant Infimum und $(-\infty,inf(A)]$ ist die Menge aller unteren Schranken von $A$. $A$ nicht nach unten beschränkt $\Leftrightarrow inf(A)=-\infty$. $inf(\emptyset)=+\infty$.}

\COR{1.1.16}{Seien $A\subseteq B\subseteq \mathbb{R}$. Dann
\begin{enumerate}
    \item $B$ nach oben beschränkt $\Rightarrow sup(A)\leq sup(B)$.
    \item $B$ nach unten beschränkt $\Rightarrow inf(B)\leq inf(A)$.
\end{enumerate}}

\DEF{Abzählbarkeit/Kardinalität}{\begin{enumerate}
    \item Seien $X,Y$ zwei Mengen. Dann $X,Y$ gleichmächtig $\Leftrightarrow \exists$ Bijektion $f:X\rightarrow Y$.
    \item Sei $X$ eine Menge. Dann $X$ ist endlich $\Leftrightarrow (X=\emptyset) \lor (\exists n \in\mathbb{N}: f: X \rightarrow \{1,2,3,...,n\})$.
    \item Sei $X$ eine Menge. $X$ ist abzählbar (abzählbar unendlich) $\Leftrightarrow (X$ ist endlich$) \lor (X$ ist gleichmächtig wie $\mathbb{N})$.
\end{enumerate}}

\SA{1.1.20 Cantor}{$\mathbb{R}$ ist nicht abzählbar bzw. überabzählbar.}

\NOTE{}{$\lambda^2+1=0$ hat in $\mathbb{R}$ keine Lösung.}