\mysection[Aquamarine]{\centering Weiteres}
\mysubsection{Rechenregeln für Summen}
\begin{enumerate}
    \item $\sum_{i=L}^Ux_i=\sum_{i=L}^{I-1}x_i+\sum_{i=I}^Ux_i$
    \item $\sum_{i=L}^Ucx_i=c\sum_{i=L}^Ux_i$
    \item $\sum_{i=L}^Ux_i\pm\sum_{i=L}^Uy_i=\sum_{i=L}^U(x_i\pm y_i)$
    \item $\sum_{i=L_1}^{U_1}x_{i_1}\cdot ... \cdot \sum_{i=L_n}^{U_n}x_{i_n}=\sum_{i=L_1}^{U_1}\cdot ... \cdot \sum_{i=L_n}^{U_n}(x_{i_1}\cdot ...\cdot x_{i_n})$
    \item $\sum_{i=L_1}^{U_1}\sum_{j=L_2}^{U_2}x_{i,j}=\sum_{j=L_2}^{U_2}\sum_{i=L_1}^{U_1}x_{i,j}$
\end{enumerate}
\mysubsection{Summenformeln}
(1) $\sum_{i=1}^{n} i = 1 + 2 + 3 + ... + n = \frac{n(n+1)}{2} = \frac{1}{2}n^2 + \frac{1}{2}n$

(2) $\sum_{i=1}^{n} i^2 = 1^2 + 2^2 + 3^2 + ... + n^2 = \frac{n(n+1)(2n + 1)}{6} = \frac{1}{3}n^3 + \frac{1}{2}n^2 + \frac{1}{6}n$

(3) $\sum_{i=1}^{n} i^3 = 1^3 + 2^3 + 3^3 + ... + n^3 = \frac{n^2(n+1)^2}{4} = \frac{1}{4}n^4 + \frac{1}{2}n^3 + \frac{1}{4}n^2$

(4) $\sum_{k=0}^{n} q^k = q^0 + q^1 + q^2 + ... + q^n = \frac{q^{n+1}-1}{q-1}=\frac{1-q^{n+1}}{1-q}$

(5) $(j+1)^2 - j^2 = 2j + 1$

(6) $C_1 \cdot n^{k+1} \leq \sum_{i=1}^{n} i^k \leq C_2 \cdot n^{k+1}$ where $C_1 = \frac{1}{2^{k+1}}$ and $C_2 = 1$ are two constants independent of $n$. Hence, when n is large, $\sum_{i=1}^n i^k$ behaves "almost like $n^{k+1}$" up to a constant factor. In other words $\sum_{i=1}^n i^k = \theta (n^{k+1})$.

\mysubsection{Grenzwerte}
\begin{enumerate}
    \item $lim_{n\rightarrow\infty}(1 + \frac{1}{n})=1$
    \item $lim_{n\rightarrow\infty}(1 + \frac{1}{n})^b=1^b=1\ \forall b\in\mathbb{Z}$
    \item $lim_{n\rightarrow\infty}n^aq^n=0\ \forall a\in\mathbb{Z},0\leq q < 1$
    \item $lim_{n\rightarrow\infty}(\frac{1}{1+\varepsilon})^n=0\ \forall \varepsilon>0$
    \item $lim_{n\rightarrow\infty}\sqrt[n]{a}=1\ \forall a\in\mathbb{R},a>0$
    \item $lim_{n\rightarrow\infty}(1+\frac{z}{n})^n=exp(z)=e^z$
\end{enumerate}

\mysubsection{Algebraische Gesetze}
\DEF{Potenzieren}{\begin{enumerate}
    \item $a^m\cdot a^n=a^{m+n}$
    \item $\frac{a^m}{a^n}=a^{m-n}=\frac{1}{a^{n-m}}$
    \item $(a^m)^n=a^{mn}=(a^n)^m$
    \item $a^n\cdot b^n=(ab)^n$
    \item $\frac{a^n}{b^n}=(\frac{a}{b})^n, b\not =0$
    \item $a^{-n}=\frac{1}{a^n} \Leftrightarrow a^{n}=\frac{1}{a^{-n}}$
\end{enumerate}}

\DEF{Radizieren}{
\begin{enumerate}
    \item $\sqrt[n]{a}\cdot\sqrt[n]{b}=\sqrt[n]{ab}$
    \item $\frac{\sqrt[n]{a}}{\sqrt[n]{b}}=\sqrt[n]{\frac{a}{b}},b\not=0$
    \item $\sqrt[n]{\sqrt[m]{a}}=\sqrt[n\cdot m]{a}$
    \item $a^{\frac{n}{m}}=\sqrt[m]{a^n}=\frac{1}{a^{-\frac{n}{m}}}$
\end{enumerate}}

\DEF{Logarithmieren}{
\begin{enumerate}
    \item $log_ac=\frac{ln(c)}{ln(a)}$
    \item $log_a(u\cdot v)=log_au+log_av$
    \item $log_a(\frac{u}{v})=log_au-log_av\Leftrightarrow log_a(\frac{1}{v})=-log_a(v)$
    \item $log_a(b^n)=n\cdot log_ab$
    \item $y=q^x\Rightarrow x=\frac{ln(y)}{ln(q)}=log_qy\Rightarrow q^{log_qy}=y$
    \item $u^{log_av} = v^{log_au}$
\end{enumerate}}

\DEF{Weitere nützliche Eigenschaften}{
\begin{enumerate}
    \item $\frac{u}{v}\geq\frac{w}{x} \Leftrightarrow \frac{v}{u}\leq \frac{x}{w}\ \forall u,v,w,x\in\mathbb{R},u\not=0,v\not=0,w\not=0,x\not=0$.
    \item $1-x\leq e^{-x}\Leftrightarrow 1+x\leq e^x\ \forall x\in\mathbb{R}$
    \item Sei $x>0,a\in\mathbb{R}$. Dann $x^a=e^{{ln(x)}^a}=e^{ln(x)\cdot a}=exp(a\cdot ln(x))$.
\end{enumerate}
}

\mysubsection{Algebraische Methoden}
\DEF{Umkehrfunktion}{Verwenden von $log$ und $e$ für die Vereinfachung von Gleichungen mit Potenzen/Exponenten, z.B. $lim_{n\rightarrow\infty}\frac{n^{\frac{2n+3}{n+1}}}{n^2}=lim_{n\rightarrow\infty}n^{\frac{2n+3}{n+1}-2}=lim_{n\rightarrow\infty}n^{\frac{1}{n+1}}=lim_{n\rightarrow\infty}e^{ln(n^{\frac{1}{n+1}})}=lim_{n\rightarrow\infty}e^{\frac{ln(n)}{n+1}}=e^0=1$.}

\DEF{Rationalisieren des Nenners}{Elimination der Wurzel im Nenner, z.B. $\frac{1}{\sqrt{2}}=\frac{1}{\sqrt{2}}\cdot\frac{\sqrt{2}}{\sqrt{2}}=\frac{\sqrt{2}}{2}$ oder $\frac{2}{\sqrt[4]{3x^2}}=\frac{2}{\sqrt[4]{3x^2}}\cdot\frac{\sqrt[4]{3^3x^2}}{\sqrt[4]{3^3x^2}}=\frac{2\sqrt[4]{27x^2}}{\sqrt[4]{3^4x^4}}=\frac{2\sqrt[4]{27x^2}}{3|x|}$.

Wenn der Nenner aus zwei Termen besteht, muss man mit dem dem Konjugat des Nenners multiplizieren, z.B. $\frac{1}{4-\sqrt{x}}=\frac{1}{4-\sqrt{x}}\cdot\frac{4+\sqrt{x}}{4+\sqrt{x}}=\frac{4+\sqrt{x}}{4^2+4\sqrt{x}-4\sqrt{x}+x}=\frac{4+\sqrt{x}}{16-x}$.}

\DEF{Rationalisieren des Zählers}{Selbes Prinzip wie Oben, einfach angewandt auf den Zähler.}

\DEF{Partialbruchzerlegung}{}